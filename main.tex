%-------------------------
% Resume in Latex
% Author : Jake Gutierrez
% Based off of: https://github.com/sb2nov/resume
% License : MIT
%------------------------

\documentclass[letterpaper,11pt]{article}

\usepackage{latexsym}
\usepackage[empty]{fullpage}
\usepackage{titlesec}
\usepackage{marvosym}
\usepackage[usenames,dvipsnames]{color}
\usepackage{verbatim}
\usepackage{enumitem}
\usepackage[hidelinks]{hyperref}
\usepackage{fancyhdr}
\usepackage[english]{babel}
\usepackage{tabularx}
\input{glyphtounicode}


%----------FONT OPTIONS----------
% sans-serif
% \usepackage[sfdefault]{FiraSans}
% \usepackage[sfdefault]{roboto}
% \usepackage[sfdefault]{noto-sans}
% \usepackage[default]{sourcesanspro}

% serif
% \usepackage{CormorantGaramond}
% \usepackage{charter}


\pagestyle{fancy}
\fancyhf{} % clear all header and footer fields
\fancyfoot{}
\renewcommand{\headrulewidth}{0pt}
\renewcommand{\footrulewidth}{0pt}

% Adjust margins
\addtolength{\oddsidemargin}{-0.5in}
\addtolength{\evensidemargin}{-0.5in}
\addtolength{\textwidth}{1in}
\addtolength{\topmargin}{-.5in}
\addtolength{\textheight}{1.0in}

\urlstyle{same}

\raggedbottom
\raggedright
\setlength{\tabcolsep}{0in}

% Sections formatting
\titleformat{\section}{
  \vspace{-4pt}\scshape\raggedright\large
}{}{0em}{}[\color{black}\titlerule \vspace{-5pt}]

% Ensure that generate pdf is machine readable/ATS parsable
\pdfgentounicode=1

%-------------------------
% Custom commands
\newcommand{\resumeItem}[1]{
  \item\small{
    {#1 \vspace{-2pt}}
  }
}

\newcommand{\resumeSubheading}[4]{
  \vspace{-2pt}\item
    \begin{tabular*}{0.97\textwidth}[t]{l@{\extracolsep{\fill}}r}
      \textbf{#1} & #2 \\
      \textit{\small#3} & \textit{\small #4} \\
    \end{tabular*}\vspace{-7pt}
}

\newcommand{\resumeSubSubheading}[2]{
    \item
    \begin{tabular*}{0.97\textwidth}{l@{\extracolsep{\fill}}r}
      \textit{\small#1} & \textit{\small #2} \\
    \end{tabular*}\vspace{-7pt}
}

\newcommand{\resumeProjectHeading}[2]{
    \item
    \begin{tabular*}{0.97\textwidth}{l@{\extracolsep{\fill}}r}
      \small#1 & #2 \\
    \end{tabular*}\vspace{-7pt}
}

\newcommand{\resumeSubItem}[1]{\resumeItem{#1}\vspace{-4pt}}

\renewcommand\labelitemii{$\vcenter{\hbox{\tiny$\bullet$}}$}

\newcommand{\resumeSubHeadingListStart}{\begin{itemize}[leftmargin=0.15in, label={}]}
\newcommand{\resumeSubHeadingListEnd}{\end{itemize}}
\newcommand{\resumeItemListStart}{\begin{itemize}}
\newcommand{\resumeItemListEnd}{\end{itemize}\vspace{-5pt}}

%-------------------------------------------
%%%%%%  RESUME STARTS HERE  %%%%%%%%%%%%%%%%%%%%%%%%%%%%


\begin{document}

%----------HEADING----------
% \begin{tabular*}{\textwidth}{l@{\extracolsep{\fill}}r}
%   \textbf{\href{http://sourabhbajaj.com/}{\Large Sourabh Bajaj}} & Email : \href{mailto:sourabh@sourabhbajaj.com}{sourabh@sourabhbajaj.com}\\
%   \href{http://sourabhbajaj.com/}{http://www.sourabhbajaj.com} & Mobile : +1-123-456-7890 \\
% \end{tabular*}

\begin{center}
    \textbf{\Huge \scshape Aly Sultan} \\ \vspace{2pt}
    \small 978-325-1925 $|$ \small Boston, MA \\ \vspace{2pt}
    \href{mailto:sultan.a@northeastern.edu}{\underline{sultan.a@northeastern.edu}} $|$ 
    \href{https://linkedin.com/in/aly-sultan/}{\underline{linkedin.com/in/aly-sultan/}} $|$
    \href{https://github.com/asultan123}{\underline{github.com/asultan123}}
    
\end{center}


%-----------EDUCATION-----------
\section{Education}
  \resumeSubHeadingListStart
    \resumeSubheading
      {Northeastern University}{Boston, MA}
      {Ph.D. Computer Engineering}{Expected 2025}
    \resumeSubheading
      {Northeastern University}{Boston, MA}
      {M.S. Electrical and Computer Engineering}{2023}
    \resumeSubheading
      {American University in Cairo}{Cairo, Egypt}
      {B.S. Electronics \& Communication Engineering}{2019}
  \resumeSubHeadingListEnd


%-----------EXPERIENCE-----------
\section{Experience}
  \resumeSubHeadingListStart
  \resumeSubheading
  {Graduate Software Engineering Intern}{2022 -- Present}
  {System Simulation and Modeling Group, Intel}{Part Time, Remote}
  \resumeSubSubheading{AI Cost Reduction in Simics SWCI}{}
  \resumeItemListStart
  \resumeItem{Developed an AI solution to predict regression test failures based on developer source changes commited to Git}
  \resumeItem{AI solution aims to save computational resources by running test subsets more likely to fail}
  \resumeItem{Established a developer metadata collection pipeline managing up to 2000 builds per week across 2 Simics platforms using Jenkins, Splunk and GitHub's GarphQl API} 
  \resumeItem{Validated collected data using JSON schema and produced daily data health-check reports}
  \resumeItem{Trained XGBoost model on metadata collected and achieved up to a 40\% reduction in regression test compute time on Granite Rapids and Diamond Rapids Simics models with a miss rate of 5.35\% for failing tests}
  \resumeItem{Productized AI solution by creating a Pretest prediction tool piloted by 5 developers in the SSM Server team}
  \resumeItem{Shared project insights at Intel's internal AI Everywhere Conference and the S3E Tech Exchange}
  \resumeItemListEnd

  % \resumeSubSubheading{Extending the autogen framework}{}
  % \resumeItemListStart
  % % \resumeItem{Assisted colleagues from external teams in adopting the autogen framework for diverse regression test applications}
  % \resumeItem{Transitioned SDSi, Virtualization, and IP patching regression tests in GNR from Simics CLI to Python within the Simics autogen framework, enhancing their availability to new platforms}
  % \resumeItemListEnd

  \resumeSubSubheading{Secure Software Services Module (S3M) Firmware Integration Pipeline}{}
  \resumeItemListStart
    \resumeItem{Established S3M's firmware integration pipeline, achieving daily FW deliveries for S3M's Simics model}
    \resumeItem{Created a versatile Python shell library, streamlining local and remote build operations across geographically dispersed data centers}
  \resumeItemListEnd




  \resumeSubheading
    {Graduate Research Assistant}{2020 -- Present}
    {Embedded Systems Lab, Northeastern University}{Boston, MA}
    \resumeSubSubheading{Hybrid General Matrix Multiplication and Direct Convolution Architecture (HERO)}{}
    \resumeItemListStart
      \resumeItem{Developed a SystemC model for HERO, a novel matrix multiplication and convolution accelerator for DNN inference}
      \resumeItem{Introduced Self Addressable Memory (SAM) for adaptive on-chip data orchestration in HERO}
      \resumeItem{Established HERO-SIM, a PyTorch-SystemC based simulation framework for the HERO accelerator}
      \resumeItem{Evaluated HERO's efficacy on 695 DNNs, achieving up to 30X speedup and 300X energy savings over a workstation-class CPU}
      \resumeItem{Submitted HERO manuscript to DAC 2024}
    \resumeItemListEnd
    \resumeSubSubheading{Categorized Ensemble Networks for Adversarial Attack Defence (CAEN)}{}
    \resumeItemListStart
      \resumeItem{Lead an AI defense project focused on bolstering ensemble network resilience against image-based adversarial attacks}
      \resumeItem{Developed a novel training methodology combining soft labeling with dissimilar label pairing, formulated the problem as an ILP, and solved it with Gurobi}
      \resumeItem{Training methodology achieved a 1.1X increase in robust accuracy over SOTA while reducing FLOPs by 16.8\%}
      \resumeItem{Submission of CAEN manuscript to SPIE's DCS24 conference pending}
    \resumeItemListEnd


    % \resumeSubheading
    %   {Graduate Teaching Assistant}{Fall 2022}
    %   {Electrical and Computer Engineering Department, EECE 7368}{Boston, MA}
    %   \resumeItemListStart

    %   \resumeItem{Transitioned the course from SpecC to SystemC}
    %   \resumeItem{Enhanced course realism by enabling usage of the Xilinx-QEMU co-simulation environment via Docker containers}
    %   \resumeItem{Developed clear, structured lab exercises in SystemC, providing students with initial code and documentation}
    %   \resumeItemListEnd
    


\resumeSubHeadingListEnd



% %-----------PROJECTS-----------
% \section{Projects} % add neu 
%     \resumeSubHeadingListStart
%       \resumeProjectHeading
%           {\textbf{Integer Linear Program Based Scheduler for Multi-Core Processors} $|$ \emph{Python, Pyomo, Gurobi}}{2021}
%           \resumeItemListStart
%             \resumeItem{Implemented an ILP-based scheduling model to schedule applications statically on a multicore processor }
%             \resumeItem{Model was generated using python and the pyomo library and optimized with the Gurobi solver }
%             \resumeItem{Successfully generated optimal schedules for a variety of application sizes and core count configurations}
%           \resumeItemListEnd
%       \resumeProjectHeading
%           {\textbf{Accelerating Domain Design Space Exploration with CUDA} $|$ \emph{C++, CUDA}}{2020}
%           \resumeItemListStart
%             \resumeItem{Accelerated the application binding evaluation in the Domain-specific design space exploration for streaming applications algorithm developed by NEU's ESL team}
%             \resumeItem{Improved binding evaluation runtime by $\sim$100X over CPU baseline using CUDA}
%           \resumeItemListEnd
%         \resumeProjectHeading
%           {\textbf{Cache Optimization for CNN Inference} $|$ \emph{C, Darknet, Intel Pin}}{2019}
%           \resumeItemListStart
%             \resumeItem{Integrated Intel Pin with the Darknet framework to determine the effect of cache configurations on CNN inference}
%             \resumeItem{Explored effects of cache hierarchy levels, sizing and replacement policy on data movement to and from DRAM during CNN inference} % Result of that exploration
%           \resumeItemListEnd
%         \resumeProjectHeading
%           {\textbf{Darknet Convolution Inference Accelerator} $|$ \emph{C, Darknet, Vivado HLS}}{2019}
%           \resumeItemListStart
%             \resumeItem{Developed a General Matrix Multiplication (GEMM) accelerator using Vivado HLS}
%             % \resumeItem{Synthesized accelerator on a Zynq-7020 FPGA}
%             \resumeItem{Integrated accelerator into the Darknet framework}
%             \resumeItem{Accelerated inference time of the tiny darknet network by a factor of ~2X over CPU Baseline on a Zynq-7020 soc}
%           \resumeItemListEnd
%     \resumeSubHeadingListEnd

%
%-----------Publications-----------
% \section{Publications}
% \begin{itemize}[leftmargin=0.15in, label={}]
%   \small{\item{J. Zhang, \textbf{A. Sultan}, M. Zandigohar, and G. Schirner, “Generating Unified Platforms Using Multigranularity Domain DSE (MG-DmDSE) Exploiting Application Similarities,” IEEE Transactions on Computer-Aided Design of Integrated Circuits and Systems, vol. 42, no. 1. Institute of Electrical and Electronics Engineers (IEEE), pp. 280–293, Jan. 2023. doi: 10.1109/tcad.2022.3172373.}}
%   \small{\item{J. Zhang, \textbf{A. Sultan}, H. Tabkhi, and G. Schirner, “MG-DmDSE: Multi-Granularity Domain Design Space Exploration Considering Function Similarity,” 2021 Design, Automation \& Test in Europe Conference \& Exhibition (DATE). IEEE, Feb. 01, 2021. doi: 10.23919/date51398.2021.9474196.}}
%   \small{\item{\textbf{A. Sultan}, A. H. Hassan, and H. Mostafa, “A Compact Low-Power Mitchell-Based Error Tolerant Multiplier,” 2018 New Generation of CAS (NGCAS). IEEE, Nov. 2018. doi: 10.1109/ngcas.2018.8572297.}}
% \end{itemize}

    % \resumeSubHeadingListStart
    %       \resumeProjectHeading{}{}
    %       \resumeItemListStart
    %       \resumeItem{Generating Unified Platforms Using Multigranularity Domain DSE (MG-DmDSE) Exploiting Application Similarities (2022) DOI: 10.1109/TCAD.2022.3172373}
    %       \resumeItem{MG-DmDSE: Multi-Granularity Domain Design Space Exploration Considering Function Similarity (2021) DOI: 10.23919/DATE51398.2021.9474196}
    %       % \resumeProjectHeading
    %       \resumeItem{A Compact Low-Power Mitchell-Based Error Tolerant Multiplier (2018) DOI: 10.1109/NGCAS.2018.8572297
    %       }
    %       \resumeItemListEnd
    % \resumeSubHeadingListEnd

%
%-----------PROGRAMMING SKILLS-----------
\section{Technical Skills}
 \begin{itemize}[leftmargin=0.15in, label={}]
    \small{\item{
     \textbf{Languages}{: \emph{Python, C/C++, SystemC}} \\
     \textbf{Framework}{: \emph{Intel Pin, Darknet}} \\
     \textbf{Developer Tools}{: \emph{Git, Docker, Jenkins, QEMU, Simics, Gurobi}} \\
     \textbf{Libraries}{: \emph{PyTorch, Numpy, Pyomo}}
    }}
 \end{itemize}

%-------------------------------------------
\end{document}
