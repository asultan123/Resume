%-------------------------
% Resume in Latex
% Author : Jake Gutierrez
% Based off of: https://github.com/sb2nov/resume
% License : MIT
%------------------------

\documentclass[letterpaper,11pt]{article}

\usepackage{latexsym}
\usepackage[empty]{fullpage}
\usepackage{titlesec}
\usepackage{marvosym}
\usepackage[usenames,dvipsnames]{color}
\usepackage{verbatim}
\usepackage{enumitem}
\usepackage[hidelinks]{hyperref}
\usepackage{fancyhdr}
\usepackage[english]{babel}
\usepackage{tabularx}
\input{glyphtounicode}


%----------FONT OPTIONS----------
% sans-serif
% \usepackage[sfdefault]{FiraSans}
% \usepackage[sfdefault]{roboto}
% \usepackage[sfdefault]{noto-sans}
% \usepackage[default]{sourcesanspro}

% serif
% \usepackage{CormorantGaramond}
% \usepackage{charter}


\pagestyle{fancy}
\fancyhf{} % clear all header and footer fields
\fancyfoot{}
\renewcommand{\headrulewidth}{0pt}
\renewcommand{\footrulewidth}{0pt}

% Adjust margins
\addtolength{\oddsidemargin}{-0.5in}
\addtolength{\evensidemargin}{-0.5in}
\addtolength{\textwidth}{1in}
\addtolength{\topmargin}{-.5in}
\addtolength{\textheight}{1.0in}

\urlstyle{same}

\raggedbottom
\raggedright
\setlength{\tabcolsep}{0in}

% Sections formatting
\titleformat{\section}{
  \vspace{-4pt}\scshape\raggedright\large
}{}{0em}{}[\color{black}\titlerule \vspace{-5pt}]

% Ensure that generate pdf is machine readable/ATS parsable
\pdfgentounicode=1

%-------------------------
% Custom commands
\newcommand{\resumeItem}[1]{
  \item\small{
    {#1 \vspace{-2pt}}
  }
}

\newcommand{\resumeSubheading}[4]{
  \vspace{-2pt}\item
    \begin{tabular*}{0.97\textwidth}[t]{l@{\extracolsep{\fill}}r}
      \textbf{#1} & #2 \\
      \textit{\small#3} & \textit{\small #4} \\
    \end{tabular*}\vspace{-7pt}
}

\newcommand{\resumeSubSubheading}[2]{
    \item
    \begin{tabular*}{0.97\textwidth}{l@{\extracolsep{\fill}}r}
      \textit{\small#1} & \textit{\small #2} \\
    \end{tabular*}\vspace{-7pt}
}

\newcommand{\resumeProjectHeading}[2]{
    \item
    \begin{tabular*}{0.97\textwidth}{l@{\extracolsep{\fill}}r}
      \small#1 & #2 \\
    \end{tabular*}\vspace{-7pt}
}

\newcommand{\resumeSubItem}[1]{\resumeItem{#1}\vspace{-4pt}}

\renewcommand\labelitemii{$\vcenter{\hbox{\tiny$\bullet$}}$}

\newcommand{\resumeSubHeadingListStart}{\begin{itemize}[leftmargin=0.15in, label={}]}
\newcommand{\resumeSubHeadingListEnd}{\end{itemize}}
\newcommand{\resumeItemListStart}{\begin{itemize}}
\newcommand{\resumeItemListEnd}{\end{itemize}\vspace{-5pt}}

%-------------------------------------------
%%%%%%  RESUME STARTS HERE  %%%%%%%%%%%%%%%%%%%%%%%%%%%%


\begin{document}

%----------HEADING----------
% \begin{tabular*}{\textwidth}{l@{\extracolsep{\fill}}r}
%   \textbf{\href{http://sourabhbajaj.com/}{\Large Sourabh Bajaj}} & Email : \href{mailto:sourabh@sourabhbajaj.com}{sourabh@sourabhbajaj.com}\\
%   \href{http://sourabhbajaj.com/}{http://www.sourabhbajaj.com} & Mobile : +1-123-456-7890 \\
% \end{tabular*}

\begin{center}
    \textbf{\Huge \scshape Aly Sultan} \\ \vspace{2pt}
    \small 978-325-1925 $|$ \small Boston, MA \\ \vspace{2pt}
    \href{mailto:sultan.a@northeastern.edu}{\underline{sultan.a@northeastern.edu}} $|$ 
    \href{https://linkedin.com/in/aly-sultan/}{\underline{linkedin.com/in/aly-sultan/}} $|$
    \href{https://github.com/asultan123}{\underline{github.com/asultan123}}
    
\end{center}


%-----------EDUCATION-----------
\section{Education}
  \resumeSubHeadingListStart
    \resumeSubheading
      {Northeastern University}{Boston, MA}
      {Ph.D. Computer Engineering}{2019 -- Expected 2025}
    \resumeSubheading
      {American University in Cairo}{Cairo, Egypt}
      {B.S. Electronics \& Communication Engineering}{2014 -- 2019}
  \resumeSubHeadingListEnd


%-----------EXPERIENCE-----------
\section{Experience}
  \resumeSubHeadingListStart

    \resumeSubheading
      {Graduate Research Assistant}{2020 -- Present}
      {Embedded Systems Lab, Northeastern University}{Boston, MA}
      \resumeItemListStart

% Made a novel arch template for gemm and conv
% Created a novel dataflow design space exploration tool that optimizes template to library of CNN networks
% written in pytorch
% CIGAR and TEMPO
% Model arch template in SystemC
% Created HERO-T Simulation environment that scans networks and calls HERO-T Simulation
% Created novel on-chip programmable memory primitive SAMS
% Developed a layer compiler that can generate data movement descriptor for SAMs
% Supervised undergrads in   
  % Integrated model into Xilinx's QEMU+SystemC
  % Developing verilog implementation for SAMS using Xilinx's stuff
  % Exploring sched synchronization in inconsistent memory environment

        % \resumeItem{Created a novel neural network accelerator dataflow exploration methodology based on two in house tools CIGAR and TEMPO}
        % \resumeItem{Used Cnn statistIcs gATherer (CIGAR) tool to prune dataflow design space of convolution accelerators}
        \resumeItem{Developed a SystemC model of a novel Hybrid general matrix multiplication and convolution accelerator (HERO)}
        \resumeItem{Developed a Template Optimization tool (TEMPO) in python that optimizes HERO templates based on available compute resources and a target library of neural networks }
        % \resumeItem{Used accelerator TEMplate Optimizer (TEMPO) tool to optimize HERO for ~419 real world neural networks}
        % \resumeItem{Embedded the HERO model in a Python-SystemC based simulation environment (HERO-Sim) that runs a cycle accurate simulation of arbitrary neural network layers written in pytorch to estimate latency and energy consumption}
        \resumeItem{Estimated HERO's latency and energy consumption using a Python-SystemC based simulation environment (HERO-Sim)}
        \resumeItem{Created a SystemC model of a novel programmable memory primitive called Self Addressable Memory (SAM) used in statically scheduling dataflows in neural network accelerator architectures}
        \resumeItem{Implemented a SAM program compiler that generates data movement programs from convolution layer descriptions to orchestrate on-chip data movement between SAMS in HERO}
        \resumeItem{Supervised a team of three undergraduate students in:}
          \resumeItemListStart
          \resumeItem{Implementing SAMs in Verilog and synthesizing them on a Zynq-7020 based FPGA}
          \resumeItem{Integrating HERO into Xilinx's SystemC+QEMU simulation environment}
          \resumeItemListEnd
        \resumeItemListEnd
      
% -----------Multiple Positions Heading-----------
%    \resumeSubSubheading
%     {Software Engineer I}{Oct 2014 - Sep 2016}
%     \resumeItemListStart
%        \resumeItem{Apache Beam}
%          {Apache Beam is a unified model for defining both batch and streaming data-parallel processing pipelines}
%     \resumeItemListEnd
%    \resumeSubHeadingListEnd
%-------------------------------------------

    % \resumeSubheading
    %   {Graduate Research Intern}{May 2019 -- July 2019}
    %   {Southwestern University}{Georgetown, TX}
    %   \resumeItemListStart
    %     \resumeItem{Explored methods to generate video game dungeons based off of \emph{The Legend of Zelda}}
    %     \resumeItem{Developed a game in Java to test the generated dungeons}
    %     \resumeItem{Contributed 50K+ lines of code to an established codebase via Git}
    %     \resumeItem{Conducted  a human subject study to determine which video game dungeon generation technique is enjoyable}
    %     \resumeItem{Wrote an 8-page paper and gave multiple presentations on-campus}
    %     \resumeItem{Presented virtually to the World Conference on Computational Intelligence}
    %   \resumeItemListEnd

  \resumeSubHeadingListEnd

%-----------PROJECTS-----------
\section{Projects}
    \resumeSubHeadingListStart
      \resumeProjectHeading
          {\textbf{Integer Linear Program based scheduler for multi-core processors} $|$ \emph{Python, Pyomo, Gurobi}}{2021}
          \resumeItemListStart
            \resumeItem{Implemented an ILP-based scheduling model to schedule applications statically on a multicore processor }
            \resumeItem{Model was generated using python and the pyomo library and optimized with Gurobi solver }
            \resumeItem{Successfully generated optimal schedules for a variety of application sizes and core count configurations}
          \resumeItemListEnd
      % \resumeProjectHeading
      %     {\textbf{Accelerating Domain Design Space Exploration With CUDA} $|$ \emph{C++, CUDA}}{Spring 2020}
      %     \resumeItemListStart
      %       \resumeItem{Accelerated the application binding evaluation in the Domain-specific design space exploration for streaming applications algorithm developed by NEU's ESL team}
      %       \resumeItem{Improved binding evaluation runtime by ~100X over CPU baseline using CUDA}
      %     \resumeItemListEnd
        \resumeProjectHeading
          {\textbf{Cache Optimization for CNN Inference} $|$ \emph{C, Darknet, Intel Pin}}{2019}
          \resumeItemListStart
            \resumeItem{Used Intel Pin and the Darknet framework to determine the effect of cache configurations on CNN inference}
            \resumeItem{Explored effects of cache hierarchy levels, sizing and replacement policy on data movement to and from DRAM during CNN inference}
          \resumeItemListEnd
        \resumeProjectHeading
          {\textbf{Darknet Convolution Inference accelerator} $|$ \emph{C, Darknet, Vivado HLS}}{2019}
          \resumeItemListStart
            \resumeItem{Developed a General Matrix Multiplication (GEMM) accelerator using Vivado HLS}
            \resumeItem{Synthesized accelerator on a Zynq-7020 development board}
            \resumeItem{Integrated accelerator into Darknet framework}
            \resumeItem{Accelerated inference time of tiny darknet network by a factor of ~2X over CPU Baseline}
          \resumeItemListEnd
    \resumeSubHeadingListEnd

%
%-----------Publications-----------
\section{Publications}
    \resumeSubHeadingListStart
      \resumeProjectHeading
          {\textbf{A Compact Low-Power Mitchell-Based Error Tolerant Multiplier} $|$ \emph{Verilog, Matlab} }{2018}
          \resumeItemListStart
            \resumeItem{Developed a novel approxmiate multiplier architecture in Verilog and evaluated numerical accuracy in JPEG compression using MATLAB }
            \resumeItem{Design improved power-delay-product by 1.9X with only 20\% reduction in peak signal-to-noise ratio \\ in JPEG compression compared to a Xilinx Zynq-7020 DSP}
            \resumeItem{Published and presented findings at NGCAS, Malta}
          \resumeItemListEnd
    \resumeSubHeadingListEnd

%
%-----------PROGRAMMING SKILLS-----------
\section{Technical Skills}
 \begin{itemize}[leftmargin=0.15in, label={}]
    \small{\item{
     \textbf{Languages}{: Python, C/C++, SystemC, Matlab, Verilog} \\
     \textbf{Framework}{: Intel Pin, Darknet} \\
     \textbf{Developer Tools}{: Git, Docker, VS Code, Vivado, Vivado HLS, QEMU, Gurobi} \\
     \textbf{Libraries}{: PyTorch, Numpy, Matplotlib, Pyomo}
    }}
 \end{itemize}

%-------------------------------------------
\end{document}
